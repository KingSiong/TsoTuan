\documentclass{article}

\usepackage[utf8]{inputenc}
\usepackage{ctex}
\usepackage{assignpkg}
\usepackage{xeCJK}
\usepackage{amsmath, amsthm, amssymb}
\usepackage{listings,xcolor}
\usepackage{geometry} % 设置页边距
\usepackage{fontspec}
\usepackage{graphicx}
\usepackage[colorlinks]{hyperref}
\usepackage{setspace}
\usepackage{fancyhdr} % 自定义页眉页脚
\usepackage{enumerate}
\usepackage{ulem}
\usepackage{scalerel}
\usepackage{stackengine}
\usepackage{xcolor}
\usepackage{polynom}
\usepackage{algorithm}
% \usepackage{algorithmic}
\usepackage{algpseudocode}
\newcommand\showdiv[1]{\overline{\smash{\hstretch{.5}{)}\mkern-3.2mu\hstretch{.5}{)}}#1}}
\newcommand\ph[1]{\textcolor{white}{#1}}
\newcommand{\tsu}{\small\kaishu\color{brown}}

\newtheorem*{thmm}{定理}
\newtheorem{thm}{定理}[section]
\newtheorem{definition}{定义}[section]
\newtheorem{lemma}{引理}[section]
\newtheorem{corollary}{推论}[section]
\newtheorem{prop}{命题}[section]
\newtheorem{attr}{性质}[section]
\newtheorem*{prf}{证明}
\newtheorem*{lprf}{引理证明}
\newtheorem{exm}{例}[section]
\newtheorem*{sol}{解}


\linespread{1.2}

\definecolor{dkgreen}{rgb}{0,0.6,0}
\definecolor{gray}{rgb}{0.5,0.5,0.5}
\definecolor{mauve}{rgb}{0.58,0,0.82}

\pagestyle{fancy}

\lhead{\CJKfamily{kai} Xi'an JiaoTong University} %以下分别为左中右的页眉和页脚
\chead{}

\rhead{\CJKfamily{kai} 第 \thepage 頁}
\lfoot{} 
\cfoot{\thepage}
\rfoot{}
\renewcommand{\headrulewidth}{0.4pt} 
\renewcommand{\footrulewidth}{0.4pt}
%\geometry{left=2.5cm,right=3cm,top=2.5cm,bottom=2.5cm} % 页边距
\geometry{left=3.18cm,right=3.18cm,top=2.54cm,bottom=2.54cm}
\setlength{\columnsep}{30pt}

\renewcommand{\algorithmicrequire}{ \textbf{Input:}} %Use Input in the format of Algorithm
\renewcommand{\algorithmicensure}{ \textbf{Output:}} %UseOutput in the format of Algorithm

\makeatletter

\makeatother

\lstset{
    language    = c++,
    numbers     = left,
    numberstyle={                               % 设置行号格式
        \small
        \color{black}
        \fontspec{Consolas}
    },
	commentstyle = \color[RGB]{0,128,0}\bfseries, %代码注释的颜色
	keywordstyle={                              % 设置关键字格式
        \color[RGB]{40,40,255}
        \fontspec{Consolas Bold}
        \bfseries
    },
	stringstyle={                               % 设置字符串格式
        \color[RGB]{128,0,0}
        \fontspec{Consolas}
        \bfseries
    },
	basicstyle={                                % 设置代码格式
        \fontspec{Consolas}
        \small\ttfamily
    },
	emphstyle=\color[RGB]{112,64,160},          % 设置强调字格式
    breaklines=true,                            % 设置自动换行
    tabsize     = 4,
    frame       = single,%主题
    columns     = fullflexible,
    rulesepcolor = \color{red!20!green!20!blue!20}, %设置边框的颜色
    showstringspaces = false, %不显示代码字符串中间的空格标记
	escapeinside={\%*}{*)},
}


% \studentIds{计试91 施劲松}{2193512032}
% \studentNames{姓名}{学号}

\assignmentName{春秋左傳導讀}
% \assignmentNumber{4}
% \subTitle{}

\date{}


\renewcommand{\contentsname}{目錄}
\begin{document}

\makecover
\tableofcontents

\section{鄭伯克段于鄢}

\section{周鄭交質}

\section{衛石碏大義滅親}

\section{齊桓公伐楚盟屈完}

\section{宮之奇諫假道}

\section{晉驪姬之亂}

\section{子魚論戰}

\section{晉公子重耳之亡}

\section{晉楚城濮之戰}

\section{秦晉殽之戰}

\section{晉靈公不君}

\noindent{\tsu 宣公二年}

晉靈公不君,厚斂以彫牆,從臺上彈人,而觀其辟丸也,宰夫胹熊蹯不熟,殺之,寘諸畚,使婦人載以過朝,趙盾,士季,見其手,問其故,而患之,將諫,士季曰,諫而不入,則莫之繼也,會請先,不入,則子繼之,三進及溜,而後視之,曰,吾知所過矣,將改之。稽首而對曰,人誰無過,過而能改,善莫大焉。《詩》曰:『靡不有初,鮮克有終』,夫如是,則能補過者鮮矣。君能有終,則社稷之固也,豈惟群臣賴之,又曰,袞職有闕,惟仲山甫補之,能補過也。君能補過,袞不廢矣。猶不改,宣子驟諫,公患之,使鉏麑賊之,晨往,寢門闢矣,盛服將朝,尚早,坐而假寐,麑退,歎而言曰,不忘恭敬,民之主也,賊民之主,不忠,棄君之命,不信,有一於此,不如死也,觸槐而死。

秋,九月,晉侯飲趙盾酒,伏甲將攻之,其右提彌明知之,趨登曰,臣侍君宴,過三爵,非禮也,遂扶以下,公嗾夫獒焉,明搏而殺之,盾曰,棄人用犬,雖猛何爲,鬥且出,提彌明死之,初,宣子田於首山,舍于翳桑,見靈輒餓,問其病,曰,不食三日矣,食之,舍其半,問之。曰:宦三年矣,未知母之存否。今近焉,請以遺之,使盡之,而爲之簞食與肉,寘諸橐以與之,既而與爲公介,倒戟以禦公徒,而免之,問何故,對曰,翳桑之餓人也。問其名居,不告而退,遂自亡也。乙丑,趙穿攻靈公於桃園,宣子未出山而復,大史書曰,趙盾弒其君,以示於朝,宣子曰,不然,對曰,子爲正卿。亡不越竟,反不討賊。非子而誰,宣子曰,嗚呼,我之懷矣,自詒伊慼,其我之謂矣。孔子曰:董狐,古之良史也。書法不隱,趙宣子,古之良大夫也,為法受惡,惜也,越竟乃免,宣子使趙穿逆公子黑臀于周,而立之,壬申,朝于武宮。

\section{楚歸晉知罃}

\noindent{\tsu 成公三年}

晉人歸楚公子\uline{穀臣},與連尹\uline{襄老}之尸于楚,以求\uline{知罃},於是\uline{荀首}{\tsu 知罃父。}佐\uline{中軍}矣,故楚人許之,\uline{王}送\uline{知罃},曰,子其怨我乎,對曰,二國治戎,臣不才,不勝其任,以為俘馘{\tsu 軍戰左截耳也,俘馘即為俘虏。},執事{\tsu 主事者,以此敬稱楚王,尊也。}不以釁{\tsu 血祭也。}鼓,使歸即戮,君之惠也,臣實不才,又誰敢怨,\uline{王}曰,然則德我乎,對曰,二國圖其社稷{\tsu 社者,土神也;稷者,穀神也。君者皆祭社稷,故引申為國。},而求紓{\tsu 緩也。}其民,各懲{\tsu 止也。}其忿,以相宥{\tsu 寬也。}也,兩釋纍{\tsu 索也。}囚,以成其好,二國有好,臣不與{\tsu 及也。}及,其誰敢德,\uline{王}曰,子歸何以報我,對曰,臣不任受怨,君亦不任受德,無怨無德,不知所報,\uline{王}曰,雖然,必告不穀{\tsu 不穀,不善也。此楚王所以自謙。},對曰,以君之靈,纍臣得歸骨於晉,寡君之以為戮,死且不朽,若從君之惠而免之,以賜君之外臣\uline{首},\uline{首}其請於寡君,而以戮於宗,亦死且不朽,若不獲命{\tsu 獲臣命,言其若不死。},而使嗣{\tsu 續也。}宗職,次及於事{\tsu 職也,前為宗職,此為國職。},而帥偏{\tsu 頗也,謙辭。}師以脩{\tsu 通「修」,飭也。}封{\tsu 疆也。}疆,雖遇執事,其弗敢違{\tsu 迴避也,謂不避楚軍。},其竭力致死,無有二心,以盡臣禮,所以報也,\uline{王}曰,晉未可與爭,重為之禮而歸之。

\section{祁奚舉賢}

\noindent{\tsu 襄公三年}

\uline{祁奚}請老{\tsu 致仕也。},\uline{晉侯}問嗣{\tsu 續也。}焉,稱{\tsu 舉也。}\uline{解狐},其讎{\tsu 仇也。《韻會》曰:「人之讎怨,不顧禮義,則如禽鳥之爲,兩怒而有言在其閒,必溢惡之言,若禽鳥之聲也。」}也,將立之而卒,又問焉。對曰,\uline{午}{\tsu 祁奚之子也。}也可。於是{\tsu 此也,於其時。}\uline{羊舌職}{\tsu 是時任中軍尉之佐。}死矣,\uline{晉侯}曰,孰可以代之。對曰,\uline{赤}{\tsu 羊舌職之子,字伯華。}也可,於是使\uline{祁午}爲\uline{中軍尉},\uline{羊舌赤}佐之,君子謂\uline{祁奚}於是能舉善{\tsu 賢也。}矣。稱其讎,不爲諂{\tsu 諂也。諂之言,陷也。},立其子,不爲比{\tsu 近也,相親也。},舉其偏{\tsu 屬也。},不爲黨{\tsu 偏也,親比也。}。\uwave{商書}曰,無偏無黨,王道蕩蕩,其\uline{祁奚}之謂矣,\uline{解狐}得舉,\uline{祁午}得位,\uline{伯華}得官,建一官而三物成{\tsu 得舉、得位、得官,此三物成所謂也。},能舉善也,夫唯善,故能舉其類,\uwave{詩}云,惟其有之,是以似之,\uline{祁奚}有焉。

\section{晏子不死君難}

\noindent{\tsu 襄公二十五年}

\uline{齊棠公}之妻,\uline{東郭偃}之姊也,\uline{東郭偃}臣\uline{崔武子}{\tsu 崔杼,諡武。},\uline{棠公}死,\uline{偃}御{\tsu 使馬也。}\uline{武子}以弔{\tsu 問終也,吊本字。}焉,見\uline{棠姜}而美之,使\uline{偃}取{\tsu 同「娶」。}之,\uline{偃}曰,男女辨{\tsu 不同也。謂東郭、崔同姓,不宜婚。}姓,今君出自\uline{丁},臣出自\uline{桓},不可,\uline{武子}筮{\tsu 占之以蓍。}之,遇困之大過{\tsu 困,象傳:「澤无水,困。君子以致命遂志。」大過,象傳:「澤滅木,大過;君子以獨立不懼,遯世无悶。」闕。},\uline{史}皆曰吉,示\uline{陳文子}{\tsu 田文子,名須無,諡文,陳公子完之後。},\uline{文子}曰,夫從風,風隕妻{\tsu 闕。},不可聚{\tsu 通「娶」。}也,且其繇{\tsu 卦兆辭也。}曰,困于石,據于蒺梨{\tsu 同「蔾」。},入于其宮,不見其妻,凶,困于石,往不濟也,據于蒺梨,可恃傷也,入于其宮,不見其妻,凶,無所歸也{\tsu 闕。},\uline{崔子}曰,嫠{\tsu 無夫也。}也何害,先夫{\tsu 謂棠公。}當{\tsu 承也,受也。}之矣,遂取{\tsu 同「娶」。}之,\uline{莊公}通{\tsu 相通而淫。}焉,驟{\tsu 數也。}如\uline{崔氏}{\tsu 邸也。},以\uline{崔子}之冠賜人,\uline{侍者}曰不可,\uline{公}曰,不為\uline{崔子},其無冠乎,\uline{崔子}因{\tsu 緣由也,由是怨公。}是,又以其間{\tsu 同「閒」,隙也。其時晉亂,齊莊公乘隙犯晉,故後言晉必報也。}伐晉也,曰晉必將報,欲弒公以說{\tsu 通「悅」。}于晉,而不獲間{\tsu 同「閒」。},\uline{公}鞭侍人\uline{賈舉},而又近之,乃為\uline{崔子}間{\tsu 同「閒」。}\uline{公},夏,五月,\uline{莒子}為\uline{且于}之役{\tsu 齊莊公伐晉未竟其功,轉而伐莒。不克。}故,\uline{莒子}朝于齊,甲戌,饗{\tsu 設禮以賓之。}諸北郭{\tsu 外城也。},\uline{崔子}稱疾不視{\tsu 治也。}事,乙亥,\uline{公}問{\tsu 慰問。}\uline{崔子},遂從{\tsu 就也。}\uline{姜氏},\uline{姜}入于室,與\uline{崔子}自側戶{\tsu 室之口也。}出,\uline{公}拊楹{\tsu 柱也。}而歌,侍人\uline{賈舉}止眾從者,而入,閉門,甲{\tsu 甲士也,借代之用。}興,公登臺而請{\tsu 求也},弗許,請盟{\tsu 締約也。},弗許,請自刃於廟,勿許,皆曰,君之臣\uline{杼}{\tsu 崔子名也。}疾病,不能聽命,近於公宮,陪臣干{\tsu 同「乾」。庶盡也乎?闕。}掫{\tsu 夜戒守,有所擊也。此為奉命巡守意,所以譏公也。}有淫{\tsu 男女不以禮交,謂之淫。}者,不知二命,公踰{\tsu 同「逾」。越也。}牆,又射之,中股,反隊{\tsu 同「墜」。落也。},遂弒之,\uline{賈舉},\uline{州綽},\uline{邴師},\uline{公孫敖},\uline{封具},\uline{鐸父},\uline{襄伊},\uline{僂堙},皆死,\uline{祝佗父}祭於\uline{高唐},至復命,不說{\tsu 通「脫」。}弁{\tsu 冕也。}而死於\uline{崔氏}{\tsu 邸也。},\uline{申蒯}侍漁者,退謂其宰{\tsu 主也。}曰,爾以帑{\tsu 財也。}免,我將死,其宰曰,免,是反子之義也,與之皆死。\uline{崔氏}殺\uline{鬷蔑}于\uline{平陰}。

\noindent{\tsu 古文觀止略為:崔武子見棠姜而美之,遂取之。莊公通焉。崔子弒之。}

\uline{晏子}{\tsu 名嬰,字仲,諡平。}立於\uline{崔氏}{\tsu 邸也。}之門{\tsu 內曰户,外曰門。堂之口也。}外,其人曰,死乎,曰,獨吾君也乎哉,吾死也?曰,行乎,曰,吾罪也乎哉,吾亡也?曰,歸乎,曰,君死安{\tsu 與「焉」同。}歸?君民者,豈以陵{\tsu 同「凌」,犯也。}民?社稷是主,臣君者,豈為其口實{\tsu 祿也。},社稷是養{\tsu 奉也。}。故君為社稷死,則死之,為社稷亡,則亡之,若為己死而己亡,非其私暱{\tsu 私也。},誰敢任{\tsu 負也。}之,且人有君而弒之,吾焉得死之,而焉得亡之,將{\tsu 且也。}庸{\tsu 用也。闕。}何歸,門啟而入,枕尸{\tsu 同「屍」。}股而哭,興{\tsu 起也。},三踊{\tsu 跳也。}而出,人謂\uline{崔子}必殺之,\uline{崔子}曰,民之望{\tsu 為人所仰者也。}也,舍{\tsu 同「捨」。}之得民。

\section{鄭子產爲政}

\section{晏嬰論和與同}

\section{伍員奔吴}

\section{齊魯夾谷之會}

\section{伍員諫許越平}

\end{document}
