\documentclass{article}

\usepackage[utf8]{inputenc}
\usepackage{ctex}
\usepackage{assignpkg}
\usepackage{xeCJK}
\usepackage{amsmath, amsthm, amssymb}
\usepackage{listings,xcolor}
\usepackage{geometry} % 设置页边距
\usepackage{fontspec}
\usepackage{graphicx}
\usepackage[colorlinks]{hyperref}
\usepackage{setspace}
\usepackage{fancyhdr} % 自定义页眉页脚
\usepackage{enumerate}
\usepackage{ulem}
\usepackage{scalerel}
\usepackage{stackengine}
\usepackage{xcolor}
\usepackage{polynom}
\usepackage{algorithm}
% \usepackage{algorithmic}
\usepackage{algpseudocode}
\newcommand\showdiv[1]{\overline{\smash{\hstretch{.5}{)}\mkern-3.2mu\hstretch{.5}{)}}#1}}
\newcommand\ph[1]{\textcolor{white}{#1}}

\newtheorem*{thmm}{定理}
\newtheorem{thm}{定理}[section]
\newtheorem{definition}{定义}[section]
\newtheorem{lemma}{引理}[section]
\newtheorem{corollary}{推论}[section]
\newtheorem{prop}{命题}[section]
\newtheorem{attr}{性质}[section]
\newtheorem*{prf}{证明}
\newtheorem*{lprf}{引理证明}
\newtheorem{exm}{例}[section]
\newtheorem*{sol}{解}


\linespread{1.2}

\definecolor{dkgreen}{rgb}{0,0.6,0}
\definecolor{gray}{rgb}{0.5,0.5,0.5}
\definecolor{mauve}{rgb}{0.58,0,0.82}

\pagestyle{fancy}

\lhead{\CJKfamily{kai} Xi'an JiaoTong University} %以下分别为左中右的页眉和页脚
\chead{}

\rhead{\CJKfamily{kai} 第 \thepage 頁}
\lfoot{} 
\cfoot{\thepage}
\rfoot{}
\renewcommand{\headrulewidth}{0.4pt} 
\renewcommand{\footrulewidth}{0.4pt}
%\geometry{left=2.5cm,right=3cm,top=2.5cm,bottom=2.5cm} % 页边距
\geometry{left=3.18cm,right=3.18cm,top=2.54cm,bottom=2.54cm}
\setlength{\columnsep}{30pt}

\renewcommand{\algorithmicrequire}{ \textbf{Input:}} %Use Input in the format of Algorithm
\renewcommand{\algorithmicensure}{ \textbf{Output:}} %UseOutput in the format of Algorithm

\makeatletter

\makeatother

\lstset{
    language    = c++,
    numbers     = left,
    numberstyle={                               % 设置行号格式
        \small
        \color{black}
        \fontspec{Consolas}
    },
	commentstyle = \color[RGB]{0,128,0}\bfseries, %代码注释的颜色
	keywordstyle={                              % 设置关键字格式
        \color[RGB]{40,40,255}
        \fontspec{Consolas Bold}
        \bfseries
    },
	stringstyle={                               % 设置字符串格式
        \color[RGB]{128,0,0}
        \fontspec{Consolas}
        \bfseries
    },
	basicstyle={                                % 设置代码格式
        \fontspec{Consolas}
        \small\ttfamily
    },
	emphstyle=\color[RGB]{112,64,160},          % 设置强调字格式
    breaklines=true,                            % 设置自动换行
    tabsize     = 4,
    frame       = single,%主题
    columns     = fullflexible,
    rulesepcolor = \color{red!20!green!20!blue!20}, %设置边框的颜色
    showstringspaces = false, %不显示代码字符串中间的空格标记
	escapeinside={\%*}{*)},
}


% \studentIds{计试91 施劲松}{2193512032}
% \studentNames{姓名}{学号}

\assignmentName{春秋左傳導讀}
% \assignmentNumber{4}
% \subTitle{}

\date{}


\renewcommand{\contentsname}{目錄}
\begin{document}

\makecover
\tableofcontents

\section{鄭伯克段于鄢}

\section{周鄭交質}

\section{衛石碏大義滅親}

\section{齊桓公伐楚盟屈完}

\section{宮之奇諫假道}

\section{晉驪姬之亂}

\section{子魚論戰}

\section{晉公子重耳之亡}

\section{晉楚城濮之戰}

\section{秦晉殽之戰}

\section{晉靈公不君}

晉靈公不君,厚斂以彫牆,從臺上彈人,而觀其辟丸也,宰夫胹熊蹯不熟,殺之,寘諸畚,使婦人載以過朝,趙盾,士季,見其手,問其故,而患之,將諫,士季曰,諫而不入,則莫之繼也,會請先,不入,則子繼之,三進及溜,而後視之,曰,吾知所過矣,將改之。稽首而對曰,人誰無過,過而能改,善莫大焉。《詩》曰:『靡不有初,鮮克有終』,夫如是,則能補過者鮮矣。君能有終,則社稷之固也,豈惟群臣賴之,又曰,袞職有闕,惟仲山甫補之,能補過也。君能補過,袞不廢矣。猶不改,宣子驟諫,公患之,使鉏麑賊之,晨往,寢門闢矣,盛服將朝,尚早,坐而假寐,麑退,歎而言曰,不忘恭敬,民之主也,賊民之主,不忠,棄君之命,不信,有一於此,不如死也,觸槐而死。

秋,九月,晉侯飲趙盾酒,伏甲將攻之,其右提彌明知之,趨登曰,臣侍君宴,過三爵,非禮也,遂扶以下,公嗾夫獒焉,明搏而殺之,盾曰,棄人用犬,雖猛何為,鬥且出,提彌明死之,初,宣子田於首山,舍于翳桑,見靈輒餓,問其病,曰,不食三日矣,食之,舍其半,問之。曰:宦三年矣,未知母之存否。今近焉,請以遺之,使盡之,而為之簞食與肉,寘諸橐以與之,既而與為公介,倒戟以禦公徒,而免之,問何故,對曰,翳桑之餓人也。問其名居,不告而退,遂自亡也。乙丑,趙穿攻靈公於桃園,宣子未出山而復,大史書曰,趙盾弒其君,以示於朝,宣子曰,不然,對曰,子為正卿。亡不越竟,反不討賊。非子而誰,宣子曰,嗚呼,我之懷矣,自詒伊慼,其我之謂矣。孔子曰:董狐,古之良史也。書法不隱,趙宣子,古之良大夫也,為法受惡,惜也,越竟乃免,宣子使趙穿逆公子黑臀于周,而立之,壬申,朝于武宮。

\section{楚歸晉知罃}

晉人歸楚公子\underline{穀臣},與連尹\underline{襄老}之尸于楚,以求\underline{知罃},於是\underline{荀首}佐中軍矣,故楚人許之,王送知罃,曰,子其怨我乎,對曰,二國治戎,臣不才,不勝其任,以為俘馘,執事不以釁鼓,使歸即戮,君之惠也,臣實不才,又誰敢怨,王曰,然則德我乎,對曰,二國圖其社稷,而求紓其民,各懲其忿,以相宥也,兩釋纍囚,以成其好,二國有好,臣不與及,其誰敢德,王曰,子歸何以報我,對曰,臣不任受怨,君亦不任受德,無怨無德,不知所報,王曰,雖然,必告不穀,對曰,以君之靈,纍臣得歸骨於晉,寡君之以為戮,死且不朽,若從君之惠而免之,以賜君之外臣首,首其請於寡君,而以戮於宗,亦死且不朽,若不獲命,而使嗣宗職,次及於事,而帥偏師以脩封疆,雖遇執事,其弗敢違,其竭力致死,無有二心,以盡臣禮,所以報也,王曰,晉未可與爭,重為之禮而歸之。

\section{祁奚舉賢}

	
祁奚請老,晉侯問嗣焉,稱解狐,其讎也,將立之而卒,又問焉。對曰,午也可。於是羊舌職死矣,晉侯曰,孰可以代之。對曰,赤也可,於是使祁午為中軍尉,羊舌赤佐之,君子謂祁奚於是能舉善矣。稱其讎,不為諂,立其子,不為比,舉其偏,不為黨。商書曰,無偏無黨,王道蕩蕩,其祁奚之謂矣,解狐得舉,祁午得位,伯華得官,建一官而三物成,能舉善也,夫唯善,故能舉其類,詩云,惟其有之,是以似之,祁奚有焉。

\section{晏子不死君難}

\section{鄭子產爲政}

\section{晏嬰論和與同}

\section{伍員奔吴}

\section{齊魯夾谷之會}

\section{伍員諫許越平}

\end{document}
