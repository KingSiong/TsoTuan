\documentclass{article}

\usepackage[utf8]{inputenc}
\usepackage{ctex}
\usepackage{assignpkg}
\usepackage{xeCJK}
\usepackage{amsmath, amsthm, amssymb}
\usepackage{listings,xcolor}
\usepackage{geometry} % 设置页边距
\usepackage{fontspec}
\usepackage{graphicx}
\usepackage[colorlinks]{hyperref}
\usepackage{setspace}
\usepackage{fancyhdr} % 自定义页眉页脚
\usepackage{enumerate}
\usepackage{ulem}
\usepackage{scalerel}
\usepackage{stackengine}
\usepackage{xcolor}
\usepackage{polynom}
\usepackage{algorithm}
% \usepackage{algorithmic}
\usepackage{algpseudocode}
\newcommand\showdiv[1]{\overline{\smash{\hstretch{.5}{)}\mkern-3.2mu\hstretch{.5}{)}}#1}}
\newcommand\ph[1]{\textcolor{white}{#1}}



\newtheorem*{thmm}{定理}
\newtheorem{thm}{定理}[section]
\newtheorem{definition}{定义}[section]
\newtheorem{lemma}{引理}[section]
\newtheorem{corollary}{推论}[section]
\newtheorem{prop}{命题}[section]
\newtheorem{attr}{性质}[section]
\newtheorem*{prf}{证明}
\newtheorem*{lprf}{引理证明}
\newtheorem{exm}{例}[section]
\newtheorem*{sol}{解}


\linespread{1.2}

\definecolor{dkgreen}{rgb}{0,0.6,0}
\definecolor{gray}{rgb}{0.5,0.5,0.5}
\definecolor{mauve}{rgb}{0.58,0,0.82}

\pagestyle{fancy}

\lhead{\CJKfamily{kai} Xi'an JiaoTong University} %以下分别为左中右的页眉和页脚
\chead{}

\rhead{\CJKfamily{kai} 第 \thepage 页}
\lfoot{} 
\cfoot{\thepage}
\rfoot{}
\renewcommand{\headrulewidth}{0.4pt} 
\renewcommand{\footrulewidth}{0.4pt}
%\geometry{left=2.5cm,right=3cm,top=2.5cm,bottom=2.5cm} % 页边距
\geometry{left=3.18cm,right=3.18cm,top=2.54cm,bottom=2.54cm}
\setlength{\columnsep}{30pt}

\renewcommand{\algorithmicrequire}{ \textbf{Input:}} %Use Input in the format of Algorithm
\renewcommand{\algorithmicensure}{ \textbf{Output:}} %UseOutput in the format of Algorithm

\makeatletter

\makeatother

\lstset{
    language    = c++,
    numbers     = left,
    numberstyle={                               % 设置行号格式
        \small
        \color{black}
        \fontspec{Consolas}
    },
	commentstyle = \color[RGB]{0,128,0}\bfseries, %代码注释的颜色
	keywordstyle={                              % 设置关键字格式
        \color[RGB]{40,40,255}
        \fontspec{Consolas Bold}
        \bfseries
    },
	stringstyle={                               % 设置字符串格式
        \color[RGB]{128,0,0}
        \fontspec{Consolas}
        \bfseries
    },
	basicstyle={                                % 设置代码格式
        \fontspec{Consolas}
        \small\ttfamily
    },
	emphstyle=\color[RGB]{112,64,160},          % 设置强调字格式
    breaklines=true,                            % 设置自动换行
    tabsize     = 4,
    frame       = single,%主题
    columns     = fullflexible,
    rulesepcolor = \color{red!20!green!20!blue!20}, %设置边框的颜色
    showstringspaces = false, %不显示代码字符串中间的空格标记
	escapeinside={\%*}{*)},
}


% \studentIds{计试91 施劲松}{2193512032}
% \studentNames{姓名}{学号}

\assignmentName{晋楚城濮之战}
% \assignmentNumber{4}
\subTitle{讨论报告}

\date{}

\begin{document}

\makecover
\tableofcontents

\section{序}

中华文化之盛,皆起于春秋。先秦诸子之文,百读犹有所得,《春秋左传》为其一。十组组员据僖公二十七至二十八年所载,而晓城濮之战。期于此日,會于一堂,喻己说而采人之长见。众人之说各有其理,辞无不畅达,余故为之录,编此报告。作序。

伯者,霸也,诸侯皆所欲图也。晋重耳之出亡一十九年,上国不遇,野人不畏,苦难备尝矣。花甲乃立,雄心未泯。故以四年之教成民礼,以三舍之避成君德。于是得齐秦之力,战于城濮,克楚师而盟践土,威风八面,其志得矣。

然一战成霸,未几而薨。后楚庄王立,三年而一飞冲天,寻问鼎于中原,飲马于黄河,败晋于两棠。输赢谁料,而其伯业尚存乎?昔齐桓尊王攘夷,伐戎破狄,三存亡国,而问封禅于葵丘,非得势乎?然盛极而衰,身名竟归于尘土,唏嘘未了,可以伯业称乎?

说文曰:「王者,天下所归往也。」伯者近然。周公捉发吐哺,于是天下归心。周公去,而天下渐背。德配其位,人乃服,人服则同心而戮力,盈盈无数也。不然,利己而害人,盈必致亏。今谓之零和博弈,期望无有胜者。晋文公溃楚师为己利,非为德也;盟践土畏诸侯,非同心也。五霸皆为一时之霸阙人而谋己利,竟致于衰,岂非定数哉?

\section{叙事}

左氏之叙事,条理井然。楚欲霸而围宋,子文致政于子玉,宋告急于晋。晋以谋略,伐曹卫而救宋,更立其威。至此,人物聚齐。

而后晋楚各施其计,亦相当精彩。曹卫既克,而楚师不去宋,宋再使告急,晋文公为之难。先轸献谋,将宋赂之于齐秦,假他人手以告楚。而为防楚师返而霸止去,又以曹卫之地畀宋,绝楚后路,是为绝计。晋有谋,楚亦有计。宛春如晋军,告以封曹人复卫则释宋,看似两全之策,实则陷晋于两难。然先轸再献计,私许曹卫以复国而绝楚,拘宛春以激子玉,势在必得。可见晋于谋略更胜一筹。左氏之叙,有辩有证,详略有当,令人叹服。

退避三舍,次于城濮,大战在即,一切就绪。故事直迫高潮,至此左氏再述舆人之奇诵、战前公之异梦,而以子犯风趣之言解之,气氛稍缓,又为一绝。战前再提文公、子玉君臣战书相与,子玉骄之甚于此可见一斑,胜负昭然。奇计致胜,盟于践土,亦详尽。左氏之笔,叙述备矣。

\section{人物}

\subsection{晋文公}

晋文公,城濮之战之核心。于文中可见其谨慎稳重、知人善任。伐曹之时,曹人尸诸城上;城濮之战前,舆人奇诵:「原田每每,舍其旧而是新谋。」此皆细末,而晋侯觉察,恐军心为之动,可见慎之又慎。顛頡、魏犨有十九年出亡之从,违令而火烧僖负羁之邸,而杀顛頡以徇,残忍且过分稳重。知人善任,有先轸、子犯、赵衰、郤縠之属,兼听而善用,故可威于诸侯。

\subsection{楚成王}

楚成王,杀兄即位,亦谨慎者。知晋文公能任事,欲罢子玉之兵,不听,则少与之师。子玉之败,其无责乎?

\subsection{成得臣}

成得臣,字子玉,楚国令尹。治军有方,易怒。子文,能人也,其致政于子玉,可见子玉亦非常人。然过刚而折,令人叹息。可败晋者,子玉也;败于晋者,亦子玉也。

\subsection{狐偃}

狐偃,字子犯,晋文公舅氏。十九年从亡,忠之属。文中特点为善言,公多次疑虑不进,总能以言说,或多风趣。如晋侯异梦,晋侯仰天倒地,曰:「我得天」;楚王伏其身,曰:「伏其罪」;盬其脑,曰:「以柔克刚」。神乎其技,出亡时,野人与之土块,子犯亦言:「天赐也」。可比管仲之识俞儿。

\subsection{先轸}

足智多谋者。先时,宋告急,献策伐曹卫救宋,一举两得。宋再告,并与赂,先轸以赂齐秦,以曹卫地畀宋,化难为易。拘宛春而私许曹卫复国,激子玉,有此略,谁堪伯仲?城濮之战亦有其谋,文中未及。

\section{论辩}

\subsection*{题}

晋文公何以胜楚。

\subsection*{解}

晋之克楚,在于谋,在于变。

谋在于何?楚之围宋,先轸知此为图霸之时,伐卫曹以激楚。楚王患,欲去宋之兵,子玉不从,已陷之。宛春之计本善,然其技逊于先轸,子玉骄怒而从晋,己将死矣。欲胜于城濮,难甚。

变在于何?曹人陈尸城上,迁师于曹墓,反使曹患。楚师相逼,反以退迎,更激子玉而成君仁。次于城濮,楚舍于险,晋军故伐木益兵,蒙马以虎皮,化从为主,如何不克?

\subsection*{题}

子曰:「晋文公谲而不正,齐桓公正而不谲。」何解。

\subsection*{解}

子玉,勇猛之士且治军之能才。若欲克晋,亦子玉也。然子玉兵败于城濮,自刎于连谷,盖晋之谋盛而子玉骄甚矣。此谓之晋文公谲。

齐桓公拜管仲为相,尊王而攘夷,伐山戎而定孤竹,救邢伐狄,实至名归。此谓之齐桓公正。然救邢之时,信管仲之言,以逸待邢溃而狄疲,方克,亦谲也。谓齐正于晋,诚然,谓齐正,不尽然。正者方能长存,是时礼崩乐坏,谲者生而正者灭。齐桓公以尊王攘夷而霸,犹为可信,晋文公虽言尊王,实为晋利,及至楚庄王,莫有辞焉。


\end{document}
